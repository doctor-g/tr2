\documentclass[letterpaper]{article}
\usepackage{url}
\usepackage{natbib}

\newcommand{\serc}{S$^2$ERC}

\title{What a Cybersecurity Education Game Reveals About How Children Conceive of Security and Software Engineering\\
\medskip
\serc{} Technical Report \textbf{NNN}\\
}

\author{Paul Gestwicki and Kaleb Stumbaugh\\
Computer Science Department\\
Ball State University\\
Muncie, IN 47306\\
Email: pvgestwicki@bsu.edu}

\begin{document}

\maketitle


\begin{abstract}
  Abstract goes here.
\end{abstract}

\section{Introduction}

The Cybersecurity Education Workshop Final
Report~\cite{Cybersecurity2014} identified six major research themes
for immediate action, in order to address deficiencies in contemporary
computer security education practice. The first of these---\textit{Concepts and Conceptual Understanding}---was the topic of our earlier work
on cybersecurity epistemology, particularly as applied to
educational games~\citep{Gestwicki2015,Gestwicki2015-tr}.
Our analysis of the literature and a survey of the state of the art
contributed to a three-tier taxonomy for cybersecurity education games:

\begin{description}
\item[Type 1] Games that convey cybersecurity concepts through
 narrative and/or theme only. There is no representation of
 the concepts within actual gameplay. That is, the act of playing
 the game does not require any decision-making that would reflect
 an understanding of cybersecurity concepts.
\item[Type 2] Games that integrate multiple-choice questions
  (including yes/no options and branching narratives)
  that correspond to cybersecurity concepts.
  Answering these prompts correctly requires an understanding
  of the concepts.
\item[Type 3] Games that require ambiguous decision-making
 such that making good decisions implies an understanding of
 cybersecurity concepts.
 \end{description}

In this report, we turn to two additional themes from the
Cybersecurity Education Workshop Final Report:
\textit{Assessment} and \textit{Recruitment and Retention}.
The former deals with the development of reliable and valid
techniques for measuring subjects' understanding of cybersecurity
as well as methods for evaluating educational interventions.
The latter deals with the ``pipeline'' problem, that there are not enough
people preparing for and staying in cybersecurity careers to meet the demand.

This work focuses on youth in middle school and early high school. Children at
this age make critical decisions about their futures, particularly with regard
to whether or not they can succeed as scientists 
and engineers. \citet{Margolis2003} and \citet{Margolis2010} describe
how cultural factors disproportionately affect underrepresented groups,
contributing to their absence in IT careers generally.
Considering this population, we identified the following research questions:
\begin{itemize}
\item[RQ1] What do youth think about software development?
\item[RQ2] What do youth think about education and careers in technology?
\item[RQ3] What do youth think about security?
\item[RQ4] How does an educational game impact these?
\end{itemize}

To address these questions, we embarked on an iterative and incremental
approach to game design and development, following best practices
of agile game software development.
The result of this process is \textit{The Social Startup Game}, 
a Type~3 cybersecurity education game that explores themes of
technology careers, educational paths, and security and
software engineering fundamentals.
Section~\ref{sec:design} of this report documents our design process
and results. Section~\ref{sec:evaluation} describes our
qualitative evaluation protocol and results, and Section~\ref{sec:discussion}
contextualizes these results within the overarching research framework.


\section{Game Design}
\label{sec:design}

\textit{Kaleb will provide text here describing the game design}

\section{Evaluation}
\label{sec:evaluation}

To answer our research questions requires building an understanding of
the player from descriptions and observations, and so we follow a
qualitative methodology~\citep[see][for example]{Stake2010}.  The
paucity of cybersecurity educational game assessments leaves almost no
grounds for forming the reliable and valid instruments necessary for
quantitative methods. Put another way, we could not know whether what
we measured would actually contribute to answering our research
questions.  \citeauthor{Cohen1994}'s classic essay highlights several
methodological problems with applying statistical methods to human
behavior, not the least of which is the common confusion of
\textit{probability of data} with \textit{probability of hypothesis
  correctness}~\citep{Cohen1994}.
Qualitative research methods, by contrast, deal with 
observations and descriptions, meeting the subjects within their
complex cultural contexts, and seeking understanding of phenomena
that are not directly measurable. 

...

We proceeded to code the data following the techniques described
by \citet{Saldana2009}.


\section{Discussion}
\label{sec:discussion}

\section{Conclusions and Future Work}

\section{Acknowledgments}

This research is based upon work supported by the National Science
Foundation under Grant No.~0968959 and by the Ball State University
Honors College.


\bibliographystyle{plainnat}
\bibliography{references}

\end{document}