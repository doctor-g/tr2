\documentclass[letterpaper]{article}
\usepackage{url}
\usepackage{natbib}
\usepackage{multicol}
\usepackage{fullpage}
\usepackage{mdwlist}

\newcommand{\serc}{S$^2$ERC}

\title{What a Cybersecurity Education Game Reveals About How Children Conceive of Security and Software Engineering\\
\medskip
\serc{} Technical Report \textbf{NNN}\\
}

\author{Paul Gestwicki and Kaleb Stumbaugh\\
Computer Science Department\\
Ball State University\\
Muncie, IN 47306\\
Email: pvgestwicki@bsu.edu}

\begin{document}

\maketitle


\begin{abstract}
  Abstract goes here.
\end{abstract}

\section{Introduction}

The Cybersecurity Education Workshop Final
Report~\cite{Cybersecurity2014} identified six major research themes
for immediate action, in order to address deficiencies in contemporary
computer security education practice. The first of these---\textit{Concepts and Conceptual Understanding}---was the topic of our earlier work
on cybersecurity epistemology, particularly as applied to
educational games~\citep{Gestwicki2015,Gestwicki2015-tr}.
Our analysis of the literature and a survey of the state of the art
contributed to a three-tier taxonomy for cybersecurity education games:

\begin{description}
\item[Type 1] Games that convey cybersecurity concepts through
 narrative and/or theme only. There is no representation of
 the concepts within actual gameplay. That is, the act of playing
 the game does not require any decision-making that would reflect
 an understanding of cybersecurity concepts.
\item[Type 2] Games that integrate multiple-choice questions
  (including yes/no options and branching narratives)
  that correspond to cybersecurity concepts.
  Answering these prompts correctly requires an understanding
  of the concepts.
\item[Type 3] Games that require ambiguous decision-making
 such that making good decisions implies an understanding of
 cybersecurity concepts.
 \end{description}

In this report, we turn to two additional themes from the
Cybersecurity Education Workshop Final Report:
\textit{Assessment} and \textit{Recruitment and Retention}.
The former deals with the development of reliable and valid
techniques for measuring subjects' understanding of cybersecurity
as well as methods for evaluating educational interventions.
The latter deals with the ``pipeline'' problem, that there are not enough
people preparing for and staying in cybersecurity careers to meet the demand.

This work focuses on youth in middle school and early high school. Children at
this age make critical decisions about their futures, particularly with regard
to whether or not they can succeed as scientists 
and engineers. \citet{Margolis2003} and \citet{Margolis2010} describe
how cultural factors disproportionately affect underrepresented groups,
contributing to their absence in IT careers generally.
Considering this population, we identified the following research questions:
\begin{itemize}
\item[RQ1] What do youth think about software development?
\item[RQ2] What do youth think about education and careers in technology?
\item[RQ3] What do youth think about security?
\item[RQ4] How does an educational game impact these?
\end{itemize}

To address these questions, we embarked on an iterative and incremental
approach to game design and development, following best practices
of agile game software development.
The result of this process is \textit{The Social Startup Game}, 
a Type~3 cybersecurity education game that explores themes of
technology careers, educational paths, and security and
software engineering fundamentals.
In particular, we designed and evaluated a game designed around
two learning outcomes: that the player would learn fundamentals of
cybersecurity, and that the player would learn about careers and
career paths in cybersecurity.
Section~\ref{sec:design} of this report documents our design process
and results. Section~\ref{sec:evaluation} describes our
qualitative evaluation protocol and results, and Section~\ref{sec:discussion}
contextualizes these results within the overarching research framework.


\section{Game Design}
\label{sec:design}

\textit{Kaleb will provide text here describing the game design}

\section{Evaluation}
\label{sec:evaluation}

\subsection{Methods}

To answer our research questions requires building an understanding of
the player from descriptions and observations, and so we follow a
qualitative methodology~\citep[see][for example]{Stake2010}.  The
paucity of cybersecurity educational game assessments leaves almost no
grounds for forming the reliable and valid instruments necessary for
quantitative methods. Put another way, we could not know whether what
we measured would actually contribute to answering our research
questions.  \citeauthor{Cohen1994}'s classic essay highlights several
methodological problems with applying statistical methods to human
behavior, not the least of which is the common confusion of
\textit{probability of data} with \textit{probability of hypothesis
  correctness}~\citep{Cohen1994}.
Qualitative research methods, by contrast, deal with 
observations and descriptions, meeting the subjects within their
complex cultural contexts, and seeking understanding of phenomena
that are not directly measurable. 

Our research protocol was approved by Ball State University
Institutional Review Board~(835177-1). 
Our subject pool consisted of thirteen youth, ages ten to fifteen,
with eight females and five males. All of the subjects were
white American students living in Indiana. 
They came from families who responded to a call for participation that
was sent to Ball State University mailing lists and shared via social
media. This group is a convenience sample of those who responded to the
call, chosen for their ability to schedule sessions and with an emphasis
on female participants. The subjects include a mixture of homeschool
and public school attendees.

\begin{table}
\centering
\begin{tabular}{llr}
\textbf{Subject ID} & \textbf{Gender} & \textbf{Age}\\
1 & Female & 14\\
2 & Male & 10\\
3 & Female & 12\\
4 & Female & 11\\
5 & Male & 12\\
6 & Male & 15\\
7 & Female & 12\\
8 & Male & \textbf{??}\\
9 & Female & \textbf{??}\\
10 & Female & \textbf{??}\\
11 & Male & 15\\
12 & Female & 10\\
13 & Female & 10\\
\end{tabular}
\caption{Research Subjects}
\end{table}

Each session consisted of a semi-structured interview combined with
playtesting~(Appendix~\ref{appendix:interview-protocol}.
Interview audio was digitally recorded, and these were transcribed
into 918 paragraph-separated units. 
The investigators also 
produced~27 paragraphs of analytic memos,
% PVG 1140, KCS 229
consisting of~1378 words.
Gameplay behavior was automatically tracked using integrated logging, 
producing~16 logs consisting of~326 discrete game events.

Interview data were coded using techniques inspired by
\citet{Saldana2009}. The two researchers conducted three phases 
of coding before comparing results in order to ensure inter-rater
reliability. The first round consisted of open coding, and these
codes were refined in the second round. 
Appendix~\ref{appendix:phase-2} lists our second phase codes,
consolidated and standardized between researchers.
The third round consisted of
identifying categories, which represented the concepts that emerged
out of multiple second-round codes; the categories are listed in
Appendix~\ref{appendix:categories}.

\subsection{Findings}

Our analysis resulted in four major themes, which are listed
in Table~\ref{tab:themes}.
%
\begin{table}
\begin{itemize}
\item Mixed views on education and career goals
\item Diverse opinions about developers appearance and interests
\item Impact of background knowledge
\item Two modes of character-driven decisions
\end{itemize}
\caption{Four Major Themes}
\label{tab:themes}
\end{table}
%
Each of the themes identified by our analysis are described in turn below.


\newcommand{\theme}[1]{\subsubsection*{#1}}

\theme{Mixed views on the role of education and degree toward career goals}

The subjects articulated a predictable mix of opinions about their
school experience. Although the interview question about favorite
school subjects was intended as more of an ice-breaker than anything,
many students eagerly described what subjects they enjoyed and why.
For example, subject~7 enjoyed English and science because they allow her
to be ``more creative than in other classes,'' and subject~5 likes
mathematics because identifies as ``a problem-solver.''
Subject~6 gives his favorite subject as lunch, since he is ``not a big
fan of school\ldots{} I know I have to do it, but it's not very fun.''
We observe some subjects making distinctions between ``academic'' and 
``non-academic'' courses: subject~9 prefers art and band, but she identifies
English has her favorite academic course. We lack data to indicate whether
this distinction is from schooling structures, environmental impacts (her
mother is a teacher), or deeper knowledge of epistemology.

Our subjects showed a general understanding that mathematics is important
to being a developer. According to subject~11, 
``There's probably a lot of mathematics and, you know, how you have to
know how games work or what other people did to make the games.''
% 
They also perceived Computer Science as being a default path into
application development.  Subject~6 says, ``It seems like making an
app would be more computers and that stuff than anything else.''
During the in-game event in which players have to choose what major to recommend
to a coworker's daughter, ten chose Computer Science, two chose 
Graphic Design, and one chose Engineering---a surprising consistency
of choices given that there were nine options (Table~\ref{tab:child-advice}).
%
\begin{table}
\centering
\begin{tabular}{rlrl}
\textbf{Option} & \textbf{Times Chosen}\\
Computer Science & 10\\
Engineering & 1\\
English & 0\\
Graphic Design & 2\\
Mathematics & 0\\
Marketing & 0\\
Performing Arts & 0\\
Physics & 0\\
Psychology & 0
\end{tabular}
\caption{Possible college major choices to suggest to a coworker's daughter 
 in \textit{Social Startup Game} and the number of times they were
 selected by unique subjects during interviews}
\label{tab:child-advice}
\end{table}
%
Although almost all the students refer to Computer Science as the
subject to study to become a developer, there is no other data
to suggest that they understand what Computer Science is beyond a gateway
to application development.

Some of the subjects recognized that a college education is not the only
path to becoming a successful application developer.
Subject~5 identifies reputation as an asset and entrepreneurship as a 
possibility, saying that ``\ldots{} somebody would hear about you, and you probably need to get a degree, or be an entrepreneur and start your own app-making business.''
Subject~6 recognizes that coursework can help someone learn to make apps
but that there is a wealth of ``random sources'' from which one can learn
as well:
\begin{quote}
If somebody decides they want to make 
an app, and they know how, they sit down and do it. If they don't know how,
and they still want to do it, then depending on how complex the app is they 
either go to try to figure it out at some random source, or if it's a bigger,
more complex project, they go take classes for it, and then they make the app.
\end{quote}

Subject~4 recognized that teachers hold significant sway in students'
opportunities. When asked how developers got t heir jobs, she suggested
that ``their teachers saw their ability to do something
like that in an earlier stage, so when they were able to get
the classes in high school and through college to be able to get better
at that.''
This empowerment of authority is an interesting contrast
against the entrepreneurial view described above,
yet it reflects a real and common understanding of the trajectory
from elementary school through to a career.



\theme{Diverse opinions about developers' appearances and interests}

Most of the players claimed to have never thought about the physical
characteristics of software developers, but among those that did,
there was great variance.  Subjects 6~and 13 reported developers to be
``geeky,'' which matched our expectations based on earlier work on
perceptions of engineers (\citet{Margolis2003}, for example). Neither
subject elaborated on what characterized geekiness, and we recognize
that there are potential misunderstandings in leaving the terms used
by teenagers up to the interpretation of adults.  
By contrast,
subject~10 stated that app developers look ``like normal
people\ldots{} like anybody.''
Subject~10 went on to say that the characters in the game
``didn't look like smart people---they looked like normal people.''

We did not expect the responses that painted app developers as something
attractive and aspirational. Subject~7 describes them as ``17 to sort of
early 20s in age,'' only about five to ten years older than herself.
Subject~9 was amused by Ivar's eccentric appearance, saying, ``He looks
kinda creepy, and I love it. I like his creepy long hair with a bald head.''
This was particularly amusing to us as Computer Scientists, given that
Ivar was modeled after Bjarne Stoustrup, but we did not bring this up
in the interview.

Subject~4 brings up a distinction between what she calls ``designers''
and ``producers,'' which we believe to correspond to the business functions
of application or interface design and programming. 
When asked what app developers look like, she puts it like this:
\begin{quote}
Well, I'm pretty sure [they look] like people, they're like artists
and stuff, \ldots{} I picture the people who \textit{design} it would
be kind of like artists, kind of, messy kind of regular people, and the
people who publish it would be kind of like, \textit{official}.
\end{quote}
It is worth noting that subject~4 was one of several students who had
completed a business information technology course in the local public
school, which we know to have included an introduction to programming
and which we believe to have informed students about some
specializations in the field.

Subject~6 does not like school, but he does
like \textit{Minecraft}. When asked about the appearance of \textit{Minecraft}'s
developers, he responds, ``Probably most of them are not American,
probably a lot of them are from Japan. It's a Japanese company, isn't it?''
While other subjects who brought up \textit{Minecraft} seemed to know something
about its Swedish creator, Markus Persson (``notch''), subject~6 held
an incorrect belief that it was created by Japanese developers. 
We suspect that this belief may be due to the relatively high number
of Japanese game development companies, this does not come up in the interview.

There was only one instance of overt sexism in the data:
when asked why he sent Esteban to the security conference over
Janine or Melissa, subject~2 responded cheekily, ``He's the guy.''
No other subject referenced developers' gender in their discussion
of their perceptions, backgrounds, interests, or educational preparation.
However, we recognize that gender and racial prejudices may be
subconscious or intentionally hidden from subjects' responses.



\theme{Impact of background knowledge}

We saw players' background knowledge impact their in-game behavior
and gameplay outcomes in a variety of ways.
In a theoretical sense, this is unsurprising, given the wealth of
evidence to support constructivism---the educational theory that
individuals build their own mental models based on their experiences
and background knowledge.
However, the broad and multifaceted impact that this can have on
educational game design and evaluation makes this worth analyzing.

Most players expressed surprise at the scale and persistence
of cyberattacks. They were less clear on what the hackers
were after. Subject~2 expressed a concern that revealed a few
layers of confusion about the game, saying that perhaps the
hackers ``wanted to call the game their own.''
This shows an understanding of software copyright infringement (``piracy''),
which then shaped his rationalization for their behavior.
Although some subjects showed an understanding of the business and
economic forces around software development, none of our subjects
reflected aloud on the value of private information.

In the hacker event, most of our subjects chose to report the
crime to the police. Subject~5, however, had a different
perspective, saying this while playing the game: 
\begin{quote}
I could report it to the police, or I could task someone to try to
figure it out. I'll probably just task somebody to figure it out for now
because, I don't really feel like getting mixed up with the police,
so I'll task [someone] to do it.
\end{quote}
We did not follow up to inquire why he did not want to get ``mixed up
with the police,'' but it stands in sharp contrast to those subjects
who, seeing this option, quickly decided it was the best course of
action.  Subject~10 also chose to seek vengeance over
notifying the police---an action that results in a tremendous loss of
users---and in her post-game reflection, she regretted it:
``No, I probably shouldn't have. I probably should have let
the police handle it instead of taking it into my own hands.''

Several subjects thought that decision points had right and wrong
answers, and that their task was to choose the right one.
This matches their experience with education and conventional
educational games. Given that their interaction with us was framed in terms
of an educational game, it is not surprising that this mental model
would be at the fore.
Subject~11's explanation of how he approached decision-making in the
game could just as well be about how he took a quiz in school:
``I knew some of them would hurt me no matter what I did so [I chose] the
best possible answers. And some of them, I really didn't know what to do,
so I just guessed.''
However, we also observed some players adopting a more nuanced view,
particularly those who played more than once. Subject~9 puts it this way:
\begin{quote}
See, last time I was like, ``Oh, I'll let them know,'' and everyone
got mad. So I thought I'd just keep it a secret.
But then when I was exposed, everyone was like,
 ``Why didn't you tell us that?!''
They all got really mad and left me, which I think is what made my exposure
go up. Last time they didn't like that; this time, they didn't like it more.
\end{quote}

The players did not question whether the game was an authentic
portrayal of a social media company, although they recognized
the game as being a simulation.
Subject~5 says, ``I actually kind of liked it. I liked the way you have to try and run your company---I just kind of like that type of thing.''

Finally, we note that students' vocabulary and cultural literacy
were major factors in their ability to understand the gameplay
experience.
None of the players younger than~14 understood the term ``press release,''
although they were able to understand their options when we explained
the term.
Subject~5 demonstrated a general misunderstanding of core
gameplay terms when, a few minutes into the game, he asked,
``Are exploits and features good or bad, what are these?''
Game literacy also played an important role, as it appears many of
our subjects did not recognize the character skill levels for what they were.
By contrast, upper-division high school players had no trouble recognizing
these, and we can assume that older players would have broader game literacy.


\theme{Different modes of character empathy}

The players often referenced their workers when making decisions. Several
took advantage of the game's paused state to review the bios of their
employees. We observed a dichotomy in how this information was used.
Those who engaged in \textit{pragmatic} decision-making chose
characters who were deemed to be best at these roles. This is exemplified
in subject~4's approach to dealing with the hacker event:
``They needed somebody to fix it, so
I sent [Vani] because she was good at fixing that kind
of thing.'' Similarly, subject~7 explains why sent Jerry to investigate
the injection attack, contrasting his background against Janine's:
``He seemed like the most well-trained out of all of them,
and the girl---I forget her name---she seemed more like someone
who should be with, like, social, and not developer,
but Jerry seemed like the better person for the job of that kind of thing.''
Sometimes this pragmatic approach included assumptions about 
educational background as well, as we can see in subject~11's 
explanation of why he referenced their bios so often:
\begin{quote}
I thought there was probably a pretty balanced team, because you had
someone who majored in three of the important categories,
such as computer engineering.
I can't remember what the one in the middle was, and
then there was another one who was good at working with people and
finding out what they want. That's why I chose here to do the---oh no,
I chose the engineering one to teach people how to do the password thing because
he know how to make a safe password, and then I would have chosen her to
find out what people liked\ldots{}.
\end{quote}

The other mode of character-based decision-making was \textit{empathic}:
some players justified their decisions based less on a desire to
maximize output and, instead, based on a feeling of helping the
fictional characters feel more fulfilled.
Subject~9 is the best example of this---an enthusiastic player
who quickly learned the characters' names, referenced them frequently,
and seemed particularly interested in being a good leader to them.
When we asked why she chose Janine for several events, she told us, 
``From her description, she seemed really on it. She seemed really cool.
I liked her a lot.'' 
Subject~9 also commented on how much she liked Nancy, explaining that
she saw some of herself in Nancy's artwork:
\begin{quote}
She looked really sassy. And she looked like me before
I went insane, 'cause I used to have long blond hair and that was
normal. It was gross. But she was really sassy, so I liked her.
`Sassy,' like, not really sassy but, like, `cool.' Like, she looked hip
and with the kids.'' 
\end{quote}
It is worth adding that, when subject~9 was first reading Nancy's 
educational credentials, she had other reasons to identify with
this character: Nancy and the subject's mother shared an \textit{alma mater},
and she was also impressed with Nancy's hobbies, saying, ``Nancy has a popular podcast about being a woman in technology. Oh, that's cool!'' 
Our data are not clear on whether sharing a gender contributed to this
identification as well, but we note the possibility.


\section{Discussion}
\label{sec:discussion}

\subsection{About the game}

Our findings regarding game vocabulary affirm our design decision
to avoid the terms ``confidentiality'', ``integrity'', and ``availability''
in the game. These are undeniably security jargon, not common parlance.
The subjects in our study were not a random sample of their age group,
and in fact, we have reason to believe that these were children of
privilege; we expect that players' confusion over vocabulary
and cultural references would be exaggerated among players from
less privileged backgrounds.

\textit{The Social Startup Game} features high degrees of randomization
in the spirit of increasing replayability. For example, the sequence
of story events and the set of characters included in the game are randomized,
so they will be unpredictable for any given gameplay experience.
However, we confess that while this improves replayability---generally
considered a virtue in games of any kind---it also potentially hindered
our evaluation. Each player was faced with a random set of characters,
for example, so we cannot directly compare two players' opinions of their
workers to each other, and we also could not intentionally match
or mismatch character gender and ethnicity with the player.

A significant category in our analysis included usability problems
with the game interface.
The two that dominated the list were players not realizing that they
could change characters' tasks and that they could
read character bios. This surprised us during the first few interviews,
given that all of the players read this introduction given by Frieda,
Social Jam's administrative coordinator:
\begin{quote} 
  [Employee 1], [Employee 2], and [Employee 3] are currently maintaining our
  software. You can tap them to find out more about them. You may
  reassign any number of them to new feature development at any
  time. Go ahead and try that now, and let me know when you are ready!
\end{quote}
All of the players read this and then tapped ``Next'' without ever following her
instructions, except for those whom we interrupted to demonstrate
how to do these actions.
This usability problem could likely have been discovered with 
more playtesting prior to the formal evaluation. However, we do not
believe that this had any invalidating effect upon our findings,
since players were quick to understand the interface once we
demonstrated how to use it. We suspect part of the confusion came from
the fact that our interface was designed for touch-based mobile devices,
yet our evaluation was being done on non-touch laptops with mouse
input.\footnote{The reason for using mouse-driven input was
 an untimely defect in one of our supporting libraries. 
PlayN's 2.0-rc3 contains a previously unidentified defect 
that broke popup menu support on touch-input devices, and hence player role
selection was impossible on touch-based interfaces. Although PlayN
2.0 has yet to be released, the snapshot build has fixed this defect,
and our current version of the game is built upon that snapshot.}
We observe with mouse-driven input that players scan the screen
with the pointer, watching for responses that would indicate 
interactivity, whereas on touch-based devices, players are more
apt to try tapping things to see what happens.

We used \textit{The Social Startup Game} as part of our workshop 
at the 2016 Congressional Leadership Academy, which is a full-day event
for high-achieving high school juniors from Indiana's sixth district.
The students at this event, who were just barely outside of our
target demographic, showed none of the confusion over terms
and jargon; in fact, they seemed to enjoy the game much more than our
target demographic. 
Of course, these were high-achieving students who elected
to attend a session on cybersecurity, so in a sense we were preaching
to the choir: these students were already on their way to potential
careers in cybersecurity. 

\subsection{About education}

The number and confidence of players who chose Computer Science
as the default academic option for IT careers is problematic.
While it is true that many students graduate from Computer Science
programs do pursue such careers, we know that there are many other
ways to get there. Anecdotally, we have met recruiters who prefer
liberally-educated, critical-thinking philosophy majors over 
myopic, code-oriented Computer Science majors.
If there is any standard of what constitutes Computer Science,
it is the ACM/IEEE Curriculum Guide~\citep{CS2013}---a sprawling
guide that speaks to the truth of the insider joke, ``If you ask
five computer scientists to define `computer science' you get
seven different answers.''
We are starting to see new programs being developed, including
Software Engineering programs and transdisciplinary programs like
Georgia Tech's computational media major. However, the fact remains
that to many students, computer science has ``computer'' in the title,
and it looks to them like the one surefire path toward software
development careers.

We believe there is reason to be concerned about the perception that
subjects in school are either ``academic'' or ``not academic.''
Our study does not reveal from whence these youth developed
such perspectives, although we can assume that it relates to the
administrative structure of their school experience.
The divisions of school are artificial, created by bureaucracy for
the purpose of perceived efficiency.
Students who do not recognize this as an artificial construction 
become, ironically, victims of it rather than liberally educated.

Although this study is aimed at youth, our experiences with
undergraduate computer science students revealed to us a connection
between our subjects and our university students: 
understanding of security principles and practices is negligible, 
and both pools seem to lack the concepts---the vocabulary---about
which to converse on these. This echoes the findings of the 
Cybersecurity Education Workshop~\citep{Cybersecurity2014}.
We suspect that there is a causal relationship between the
prevailing structures of Computer Science education and the lack
of good security practices in industry. The ACM/IEEE curriculum
guidelines for Computer Science curricula treat cybersecurity differently
from all other domains, explaining that this topic must be included
throughout the curriculum and not simply in isolated courses.
However, conventional higher educational structures privilege the
separation of content into courses as well as the academic freedom
of faculty in the classroom. This leads to a situation where there is
no practical way to create interleaving topics, motivate their adoption,
or evaluate their efficacy. 


\subsection{About game design}

Those who made pragmatic character-based decisions did so assuming
that character backgrounds impacted their performance, while in fact,
all of the characters are interchangeable: the differences between
them are only skin deep. 
It is interesting that the same players who lacked game literacy to
recognize certain design tropes (such as a character statistics block)
did assume that character backgrounds provided some kind of hint to
maximizing decision outcomes.
Considering two design options---one in which characters are all fundamentally
the same regardless of their appearance, bio-sketch, and credentials, and one where their in-game behavior is different based on these factors---we lack data
to determine whether one will better meet our intended learning outcomes.

The discovery of empathic character-based decisions opens more
potential opportunities for future design research.
Our prototyping process described in Section~ref{sec:design} included
several different quantitative factors such as users, defects, and money,
but none of our approaches considered emotional health, such as happiness
or job satisfaction. Players who tend toward empathic decision-making
may feel more engaged with the game by receiving such feedback, and this,
in turn, may lead to better learning outcomes.
Inspired by Bartle's classic essay~\citep{Bartle1996} in which he
describes the kinds of players who play MUDs, perhaps a similar
kind of taxonomy could be built that describes how players approach
decision-making with respect to fictional, in-game characters.
We know of no such theory, but our findings suggest that it could be
quite useful for game design generally and educational game design
specifically.


Right/wrong, education, and edugaming.


\section{Conclusions and Future Work}

kinds of players

Continued development of tier 3 games to fit this need, justified by
learner's backgrounds.

\section{Acknowledgments}

This research is based upon work supported by the National Science
Foundation under Grant No.~0968959 and by the Ball State University
Honors College.


\bibliographystyle{plainnat}
\bibliography{references}

\clearpage
\appendix
\section{Semi-structured interview protocol}
\label{appendix:interview-protocol}

\subsection*{Before playing the game}

\begin{enumerate}
\item What is your favorite subject in school?
\item What is your favorite app?
\item Who are the people who make [favorite app]?
\begin{enumerate}
\item What do they look like?
\item Did they go to college? What did they study?
\item What were their favorite subjects in school?
\end{enumerate}
\end{enumerate}

\subsection*{After playing the game}

\begin{enumerate}
\item Did anything surprise you?
\item What kinds of decisions did you make in the game? How did you make those decisions?
\item Where do apps come from?
\item Do you think you could get a job making apps? Why or why not?
\item How does someone get a job making apps?
\item Do you have any questions?
\end{enumerate}

\clearpage
\section{Second phase codes}
\label{appendix:phase-2}
\begin{multicols}{2}
\begin{verbatim}
academics: computer science
academics: science-engineering
academics: dislike

apps: communication
apps: games
apps: productivity
apps: for phones
apps: creative
apps: music

characters: pragmatic
characters: empathic

culture: male domination
culture: smart people look different

developers-like: math
developers-like: english
developers-like: art + stem
developers-like: technology
developers-like: science
developers-like: art
developers-like: design
developers-like: programming
developers-like: animation
developers-like: fundamentals
developers-like: computer science
developers-like: engineering

developers: not considered
developers: smart
developers: work hard
developers: crazy

developer-appearance: artists
developer-appearance: nonconformist
developer-appearance: geeky
developer-appearance: young
developer-appearance: smart and techy
developer-appearance: normal

development: difficulty
development: scale
development: coding
development: teams
development: testing
development: time
development: design separate from programming
development: entrepreneurship
development: anybody can do it

distribution: distinction of sources
distribution: companies as gatekeepers
distribution: appstore

economy: demand for apps
economy: jobs

education: degree has value
education: degree not necessary
education: degree as career
education: development classes
education: online resources
education: multiple paths

experience: scratch
experience: school
experience: code.org
experience: gamesalad

self-efficacy: technology requisite

game design: role-playing
game design: one right answer
game design: genre preferences
game design: empathy
game design: strategy

interest: problem-solving
interest: Minecraft
interest: creativity
interest: english
interest: science
interest: math
interest: writing
interest: terraria
interest: art
interest: band
interest: sketchbook pro
interest: history
interest: facebook
interest: wattpad
interest: gym
interest: clash of clans

learning: working hard
learning: improvement
learning: feedback loops

ssg: usability
ssg: vocabulary
ssg: math

security: disclosure
security: hackers
security: piracy
security: responsibility
security: pragmatism
security: encryption
security: confidentiality
security: number of exploits
security: strong passwords
\end{verbatim}
\end{multicols}

\clearpage
\section{Categories}
\label{appendix:categories}
\begin{itemize}
\item[] Computer science as the domain for app development
\item[] Character-based decisions
\item[] Academic vs non-academic learning
\item[] Incremental vs Entity theory of intelligence
\item[] Varied perspectives on developer appearance
\item[] Development is difficult
\item[] Companies as gatekeepers
\item[] Democracy of development
\item[] Economics and development
\item[] Degree as career
\item[] Positive educational experiences with coding
\item[] There is one right answer
\item[] Learning through practice
\item[] SSG issues
\item[] Surprise at hackers
\item[] Responsible disclosure
\item[] Protecting personal information
\item[] ``App'' means ``game''
\end{itemize}

\clearpage
\section{Characters in \textit{Social Startup Game}}

\newenvironment{character}[2]
{
\noindent\textbf{#1}
\begin{itemize*}
#2
\end{itemize*}
}
{
\bigskip
}

\begin{character}{Esteban Cortez}
{\item Bachelors in Computer Science, Ball State University}
Esteban worked in a factory until he was 33, then he went to college and decided to get involved in software development.
\end{character}

\begin{character}{Nancy Stevens}
{\item Bachelors in English, Georgetown University
 \item Masters in Computer Security, Purdue University}
Nancy has a popular podcast about being a woman in technology.
\end{character}

\begin{character}{Jerry Chen}
{\item Bachelors in Computer Science, University of Hong Kong}
Jerry interned at a local company in high school and has been working as a software developer ever since.
\end{character}

\begin{character}{Vani Mishra}
{\item Bachelors in Computer Engineering, Indian Institute of Science
 \item Masters in Software Engineering, Ball State University}
Vani was born in India and came to the United States for graduate school. She loves music, dancing, and PHP.
\end{character}

\begin{character}{Abdullah Nasr}
{\item Bachelors in Electrical Engineering, Iowa State University}
Abdullah used to work for a larger social media company, but he prefers the excitement of a small startup.
\end{character}

\begin{character}{Janine Palmer}
{\item Bachelors in Computer Science, Virginia Tech}
Janine is especially talented at meeting with customers and understanding what they want from a product.
\end{character}

\begin{character}{Melissa Barnard}
{\item Bachelors in Mathematics, Stanford University}
Melissa has always loved games and puzzles, but she especially loves bicycling and walking her German Shepherd.
\end{character}

\begin{character}{Ivar Johansen}
{\item Bachelors in Chemistry, Stockholm University
 \item Masters in Electrical Engineering, Uppsala University}
Ivar teaches kids how to build simple robots as a volunteer in a local school.
\end{character}

\begin{character}{Bruce Powers}
{\item Bachelors in Computer Science, Ball State University}
Bruce was recently married and enjoys going to the gym and watching documentaries about history.
\end{character}


\section{Gameplay Logs}
Below are the gameplay logs from each research subject, identified
with the subject's number.
When a session included more than one play through the game,
the logs are provided sequentially and annotated with a sequence number;
for example, \verb+1-2+ is subject~1's second play of the game.

\begin{verbatim}
1-1:
CEO is Nancy; workers are Janine, Vani, Abdullah, 
edu.bsu.cybersec.core.narrative.InsecurePasswordEvent: Train staff on secure passwords
edu.bsu.cybersec.core.narrative.ChildAdviceEvent: Graphic Design
edu.bsu.cybersec.core.narrative.DDOSEvent: Press Release
Set volume to zero
Set volume to zero
edu.bsu.cybersec.core.narrative.SecurityConferenceEvent: Janine
edu.bsu.cybersec.core.narrative.InputSanitizationEvent: Vani
edu.bsu.cybersec.core.narrative.DataStolenEvent: Notify our users
Final users: 2603; exposure: 0.0069495

1-2:
CEO is Melissa; workers are Esteban, Ivar, Bruce, 
Esteban task changed to Development
Esteban task changed to Maintenance
Esteban task changed to Development
Ivar task changed to Development
Ivar task changed to Development
Esteban task changed to Development
Expanded: Ivar
Collapsed: Ivar
Esteban task changed to Maintenance
Ivar task changed to Maintenance
Bruce task changed to Development
edu.bsu.cybersec.core.narrative.ChildAdviceEvent: Graphic Design
edu.bsu.cybersec.core.narrative.InputSanitizationEvent: Ivar
edu.bsu.cybersec.core.narrative.InsecurePasswordEvent: Train staff on secure passwords
Esteban task changed to Development
Esteban task changed to Maintenance
edu.bsu.cybersec.core.narrative.DDOSEvent: Just Wait
Bruce task changed to Maintenance
edu.bsu.cybersec.core.narrative.DataStolenEvent: Notify our users
Ivar task changed to Development
Esteban task changed to Development
edu.bsu.cybersec.core.narrative.SecurityConferenceEvent: Bruce
Final users: 13112; exposure: 0.1412592

2:
CEO is Abdullah; workers are Janine, Melissa, Esteban, 
Janine task changed to Development
edu.bsu.cybersec.core.narrative.DataStolenEvent: Notify our users
Esteban task changed to Development
Melissa task changed to Development
edu.bsu.cybersec.core.narrative.InputSanitizationEvent: Janine
edu.bsu.cybersec.core.narrative.InsecurePasswordEvent: Train staff on secure passwords
edu.bsu.cybersec.core.narrative.SecurityConferenceEvent: Esteban
edu.bsu.cybersec.core.narrative.ChildAdviceEvent: Computer Science
edu.bsu.cybersec.core.narrative.DDOSEvent: Press Release
Final users: 26191; exposure: 0.0674723

3:
CEO is Jerry; workers are Vani, Bruce, Nancy, 
Expanded: Nancy
Collapsed: Nancy
Expanded: Bruce
Expanded: Vani
Collapsed: Vani
Nancy task changed to Development
Bruce task changed to Development
Expanded: Vani
Collapsed: Vani
Expanded: Bruce
Collapsed: Bruce
Bruce task changed to Maintenance
Expanded: Nancy
Collapsed: Nancy
edu.bsu.cybersec.core.narrative.SecurityConferenceEvent: Nancy
Expanded: Vani
Vani task changed to Development
Vani task changed to Maintenance
Bruce task changed to Development
Collapsed: Vani
edu.bsu.cybersec.core.narrative.DDOSEvent: Press Release
edu.bsu.cybersec.core.narrative.InputSanitizationEvent: Vani
edu.bsu.cybersec.core.narrative.DataStolenEvent: Notify our users
Vani task changed to Development
Bruce task changed to Maintenance
Nancy task changed to Development
edu.bsu.cybersec.core.narrative.ChildAdviceEvent: Computer Science
edu.bsu.cybersec.core.narrative.InsecurePasswordEvent: Train staff on secure passwords
Final users: 6725; exposure: 0.092293896

4:
CEO is Esteban; workers are Janine, Nancy, Vani, 
Expanded: Janine
Collapsed: Janine
Expanded: Nancy
Collapsed: Nancy
Expanded: Vani
Collapsed: Vani
Nancy task changed to Development
Expanded: Nancy
Collapsed: Nancy
edu.bsu.cybersec.core.narrative.DataStolenEvent: Notify our users
Expanded: Janine
Expanded: Nancy
Expanded: Vani
Collapsed: Vani
edu.bsu.cybersec.core.narrative.SecurityConferenceEvent: Vani
edu.bsu.cybersec.core.narrative.ChildAdviceEvent: Graphic Design
Expanded: Janine
Expanded: Nancy
Collapsed: Nancy
edu.bsu.cybersec.core.narrative.InsecurePasswordEvent: Train staff on secure passwords
edu.bsu.cybersec.core.narrative.DDOSEvent: Press Release
Expanded: Janine
Expanded: Nancy
Expanded: Vani
Collapsed: Vani
edu.bsu.cybersec.core.narrative.InputSanitizationEvent: Nancy
Final users: 39247; exposure: 0.05003875

5:
CEO is Ivar; workers are Nancy, Vani, Melissa, 
Nancy task changed to Development
Expanded: Nancy
Expanded: Vani
Expanded: Melissa
Collapsed: Melissa
edu.bsu.cybersec.core.narrative.DDOSEvent: Press Release
Melissa task changed to Development
edu.bsu.cybersec.core.narrative.SecurityConferenceEvent: Nancy
Melissa task changed to Maintenance
edu.bsu.cybersec.core.narrative.DataStolenEvent: Notify our users
edu.bsu.cybersec.core.narrative.ChildAdviceEvent: Computer Science
Melissa task changed to Development
Melissa task changed to Maintenance
Melissa task changed to Development
edu.bsu.cybersec.core.narrative.InsecurePasswordEvent: Train staff on secure passwords
Expanded: Vani
Expanded: Nancy
Expanded: Melissa
Expanded: Vani
Collapsed: Vani
edu.bsu.cybersec.core.narrative.InputSanitizationEvent: Vani
Final users: 17423; exposure: 0.07671654

6:
CEO is Esteban; workers are Abdullah, Janine, Jerry, 
Expanded: Janine
Collapsed: Janine
Expanded: Jerry
Collapsed: Jerry
Expanded: Abdullah
Collapsed: Abdullah
Jerry task changed to Development
edu.bsu.cybersec.core.narrative.InputSanitizationEvent: Jerry
Abdullah task changed to Maintenance
Janine task changed to Development
edu.bsu.cybersec.core.narrative.DataStolenEvent: Notify our users
Jerry task changed to Development
edu.bsu.cybersec.core.narrative.SecurityConferenceEvent: Jerry
edu.bsu.cybersec.core.narrative.InsecurePasswordEvent: Train staff on secure passwords
edu.bsu.cybersec.core.narrative.ChildAdviceEvent: Computer Science
Jerry task changed to Development
edu.bsu.cybersec.core.narrative.DDOSEvent: Just Wait
Final users: 25900; exposure: 0.096139066

7:
CEO is Nancy; workers are Jerry, Esteban, Janine, 
Expanded: Jerry
Jerry task changed to Development
Collapsed: Jerry
Expanded: Esteban
Collapsed: Esteban
Expanded: Janine
Collapsed: Janine
Expanded: Esteban
Collapsed: Esteban
Expanded: Jerry
Collapsed: Jerry
edu.bsu.cybersec.core.narrative.DDOSEvent: Press Release
Esteban task changed to Development
edu.bsu.cybersec.core.narrative.SecurityConferenceEvent: Jerry
Expanded: Jerry
Collapsed: Jerry
edu.bsu.cybersec.core.narrative.ChildAdviceEvent: Engineering
edu.bsu.cybersec.core.narrative.DataStolenEvent: Notify our users
edu.bsu.cybersec.core.narrative.InsecurePasswordEvent: Train staff on secure passwords
edu.bsu.cybersec.core.narrative.InputSanitizationEvent: Jerry
Final users: 33994; exposure: 0.065729424

8:
CEO is Jerry; workers are Janine, Vani, Esteban, 
Janine task changed to Development
edu.bsu.cybersec.core.narrative.SecurityConferenceEvent: Esteban
edu.bsu.cybersec.core.narrative.DDOSEvent: Just Wait
edu.bsu.cybersec.core.narrative.InsecurePasswordEvent: Train staff on secure passwords
edu.bsu.cybersec.core.narrative.DataStolenEvent: Notify our users
edu.bsu.cybersec.core.narrative.ChildAdviceEvent: Computer Science
edu.bsu.cybersec.core.narrative.InputSanitizationEvent: Esteban
Final users: 38580; exposure: 0.070904195


9-1:
CEO is Jerry; workers are Bruce, Nancy, Janine, 
Expanded: Nancy
Expanded: Bruce
Expanded: Janine
Collapsed: Janine
edu.bsu.cybersec.core.narrative.DataStolenEvent: Notify our users
Expanded: Janine
Collapsed: Janine
Expanded: Nancy
Collapsed: Nancy
Expanded: Bruce
Collapsed: Bruce
edu.bsu.cybersec.core.narrative.DDOSEvent: Just Wait
edu.bsu.cybersec.core.narrative.ChildAdviceEvent: Computer Science
edu.bsu.cybersec.core.narrative.SecurityConferenceEvent: Bruce
edu.bsu.cybersec.core.narrative.InputSanitizationEvent: Nancy
edu.bsu.cybersec.core.narrative.InsecurePasswordEvent: Train staff on secure passwords
Final users: 5493; exposure: 0.012339515

9-2:
CEO is Melissa; workers are Ivar, Esteban, Vani, 
Expanded: Ivar
Ivar task changed to Development
Collapsed: Ivar
Vani task changed to Development
edu.bsu.cybersec.core.narrative.ChildAdviceEvent: Computer Science
edu.bsu.cybersec.core.narrative.SecurityConferenceEvent: Vani
Vani task changed to Development
edu.bsu.cybersec.core.narrative.DDOSEvent: Just Wait
edu.bsu.cybersec.core.narrative.InputSanitizationEvent: Esteban
Vani task changed to Maintenance
edu.bsu.cybersec.core.narrative.InsecurePasswordEvent: Train staff on secure passwords
Final users: 25899; exposure: 0.18964696

10:
CEO is Jerry; workers are Melissa, Bruce, Vani, 
Expanded: Melissa
Expanded: Bruce
Expanded: Vani
Expanded: Bruce
Expanded: Melissa
Expanded: Bruce
Bruce task changed to Development
Expanded: Vani
Vani task changed to Development
Collapsed: Vani
edu.bsu.cybersec.core.narrative.DataStolenEvent: Notify our users
Expanded: Melissa
Collapsed: Melissa
Expanded: Bruce
Collapsed: Bruce
Bruce task changed to Maintenance
edu.bsu.cybersec.core.narrative.ChildAdviceEvent: Computer Science
edu.bsu.cybersec.core.narrative.InputSanitizationEvent: Vani
edu.bsu.cybersec.core.narrative.DDOSEvent: Press Release
edu.bsu.cybersec.core.narrative.SecurityConferenceEvent: Melissa
edu.bsu.cybersec.core.narrative.InsecurePasswordEvent: Train staff on secure passwords
Bruce task changed to Development
Final users: 21280; exposure: 0.094213806

11-1:
CEO is Nancy; workers are Esteban, Jerry, Bruce, 
edu.bsu.cybersec.core.narrative.ChildAdviceEvent: Computer Science
Bruce task changed to Development
edu.bsu.cybersec.core.narrative.DDOSEvent: Press Release
Jerry task changed to Development
edu.bsu.cybersec.core.narrative.DataStolenEvent: Notify our users
edu.bsu.cybersec.core.narrative.InputSanitizationEvent: Esteban
edu.bsu.cybersec.core.narrative.InsecurePasswordEvent: Train staff on secure passwords
edu.bsu.cybersec.core.narrative.SecurityConferenceEvent: Jerry
Esteban task changed to Development
Jerry task changed to Maintenance
Final users: 11928; exposure: 0.15418494

11-2:
CEO is Vani; workers are Janine, Ivar, Abdullah, 
Expanded: Janine
Expanded: Ivar
Expanded: Abdullah
Collapsed: Abdullah
Abdullah task changed to Development
Expanded: Janine
Expanded: Ivar
Expanded: Abdullah
Collapsed: Abdullah
edu.bsu.cybersec.core.narrative.DataStolenEvent: Notify our users
edu.bsu.cybersec.core.narrative.InsecurePasswordEvent: Train staff on secure passwords
Expanded: Ivar
Expanded: Janine
Expanded: Abdullah
Collapsed: Abdullah
edu.bsu.cybersec.core.narrative.InputSanitizationEvent: Janine
edu.bsu.cybersec.core.narrative.ChildAdviceEvent: Computer Science
Abdullah task changed to Development
Abdullah task changed to Maintenance
Janine task changed to Development
edu.bsu.cybersec.core.narrative.DDOSEvent: Press Release
Expanded: Abdullah
Collapsed: Abdullah
edu.bsu.cybersec.core.narrative.SecurityConferenceEvent: Abdullah
Final users: 22768; exposure: 0.096434705

12: 
CEO is Abdullah; workers are Vani, Nancy, Ivar, 
Nancy task changed to Development
edu.bsu.cybersec.core.narrative.DataStolenEvent: Notify our users
Nancy task changed to Maintenance
edu.bsu.cybersec.core.narrative.SecurityConferenceEvent: Vani
Nancy task changed to Development
Ivar task changed to Maintenance
edu.bsu.cybersec.core.narrative.ChildAdviceEvent: Computer Science
Vani task changed to Development
Vani task changed to Maintenance
Ivar task changed to Development
edu.bsu.cybersec.core.narrative.InputSanitizationEvent: Ivar
Ivar task changed to Development
edu.bsu.cybersec.core.narrative.DDOSEvent: Press Release
Set volume to zero
Set volume to zero
edu.bsu.cybersec.core.narrative.InsecurePasswordEvent: Train staff on secure passwords
Final users: 31255; exposure: 0.06462003

13:
CEO is Nancy; workers are Vani, Ivar, Jerry, 
Expanded: Jerry
Jerry task changed to Development
Expanded: Ivar
Ivar task changed to Development
Collapsed: Ivar
edu.bsu.cybersec.core.narrative.ChildAdviceEvent: Computer Science
edu.bsu.cybersec.core.narrative.InputSanitizationEvent: Jerry
Jerry task changed to Development
edu.bsu.cybersec.core.narrative.DDOSEvent: Just Wait
edu.bsu.cybersec.core.narrative.InsecurePasswordEvent: Just change his password
edu.bsu.cybersec.core.narrative.SecurityConferenceEvent: Ivar
edu.bsu.cybersec.core.narrative.InsecurePasswordEvent: Train staff on secure passwords
Vani task changed to Development
Final users: 19108; exposure: 0.15697393
\end{verbatim}

\end{document}





