\documentclass[letterpaper]{article}
\usepackage{url}
\usepackage{natbib}

\newcommand{\serc}{S$^2$ERC}

\title{What a Cybersecurity Education Game Reveals About How Children Conceive of Security and Software Engineering\\
\medskip
\serc{} Technical Report \textbf{NNN}\\
}

\author{Paul Gestwicki and Kaleb Stumbaugh\\
Computer Science Department\\
Ball State University\\
Muncie, IN 47306\\
Email: pvgestwicki@bsu.edu}

\begin{document}

\maketitle


\begin{abstract}
  Abstract goes here.
\end{abstract}

\section{Introduction}

In our earlier work~\citep{Gestwicki2015,Gestwicki2015-tr},
we defined a three-tier taxonomy of cybersecurity education games:...

This leads us to identify the following research questions:
\begin{itemize}
\item What do youth think about software development?
\item What do youth think about education and careers in technology?
\item What do youth think about security?
\item How does an educational game impact these?
\end{itemize}


\section{Game Design}

\textit{Kaleb will provide text here describing the game design}

\section{Evaluation}

To answer our research questions requires building an understanding of
the player from descriptions and observations, and so we follow a
qualitative methodology~\citep[see][for example]{Stake2010}.  The
paucity of cybersecurity educational game assessments leaves almost no
grounds for forming the reliable and valid instruments necessary for
quantitative methods. Put another way, we could not know whether what
we measured would actually contribute to answering our research
questions.  \citeauthor{Cohen1994}'s classic essay highlights several
methodological problems with applying statistical methods to human
behavior, not the least of which is the common confusion of
\textit{probability of data} with \textit{probability of hypothesis
  correctness}~\citep{Cohen1994}.
Qualitative research methods, by contrast, deal with 
observations and descriptions, meeting the subjects within their
complex cultural contexts, and seeking understanding of phenomena
that are not directly measurable. 

...

We proceeded to code the data following the techniques described
by \citet{Saldana2009}.


\section{Discussion}

\section{Conclusions and Future Work}

\section{Acknowledgments}

This research is based upon work supported by the National Science
Foundation under Grant No.~0968959 and by the Ball State University
Honors College.


\bibliographystyle{plainnat}
\bibliography{references}

\end{document}