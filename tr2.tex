\documentclass[letterpaper]{article}
\usepackage{url}
\usepackage{natbib}
\usepackage{multicol}
\usepackage{fullpage}
\usepackage{mdwlist}

\newcommand{\serc}{S$^2$ERC}

\title{What a Cybersecurity Education Game Reveals About How Children Conceive of Security and Software Engineering\\
\medskip
\serc{} Technical Report \textbf{NNN}\\
}

\author{Paul Gestwicki and Kaleb Stumbaugh\\
Computer Science Department\\
Ball State University\\
Muncie, IN 47306\\
Email: pvgestwicki@bsu.edu}

\begin{document}

\maketitle


\begin{abstract}
  Abstract goes here.
\end{abstract}

\section{Introduction}

The Cybersecurity Education Workshop Final
Report~\cite{Cybersecurity2014} identified six major research themes
for immediate action, in order to address deficiencies in contemporary
computer security education practice. The first of these---\textit{Concepts and Conceptual Understanding}---was the topic of our earlier work
on cybersecurity epistemology, particularly as applied to
educational games~\citep{Gestwicki2015,Gestwicki2015-tr}.
Our analysis of the literature and a survey of the state of the art
contributed to a three-tier taxonomy for cybersecurity education games:

\begin{description}
\item[Type 1] Games that convey cybersecurity concepts through
 narrative and/or theme only. There is no representation of
 the concepts within actual gameplay. That is, the act of playing
 the game does not require any decision-making that would reflect
 an understanding of cybersecurity concepts.
\item[Type 2] Games that integrate multiple-choice questions
  (including yes/no options and branching narratives)
  that correspond to cybersecurity concepts.
  Answering these prompts correctly requires an understanding
  of the concepts.
\item[Type 3] Games that require ambiguous decision-making
 such that making good decisions implies an understanding of
 cybersecurity concepts.
 \end{description}

In this report, we turn to two additional themes from the
Cybersecurity Education Workshop Final Report:
\textit{Assessment} and \textit{Recruitment and Retention}.
The former deals with the development of reliable and valid
techniques for measuring subjects' understanding of cybersecurity
as well as methods for evaluating educational interventions.
The latter deals with the ``pipeline'' problem, that there are not enough
people preparing for and staying in cybersecurity careers to meet the demand.

This work focuses on youth in middle school and early high school. Children at
this age make critical decisions about their futures, particularly with regard
to whether or not they can succeed as scientists 
and engineers. \citet{Margolis2003} and \citet{Margolis2010} describe
how cultural factors disproportionately affect underrepresented groups,
contributing to their absence in IT careers generally.
Considering this population, we identified the following research questions:
\begin{itemize}
\item[RQ1] What do youth think about software development?
\item[RQ2] What do youth think about education and careers in technology?
\item[RQ3] What do youth think about security?
\item[RQ4] How does an educational game impact these?
\end{itemize}

To address these questions, we embarked on an iterative and incremental
approach to game design and development, following best practices
of agile game software development.
The result of this process is \textit{The Social Startup Game}, 
a Type~3 cybersecurity education game that explores themes of
technology careers, educational paths, and security and
software engineering fundamentals.
Section~\ref{sec:design} of this report documents our design process
and results. Section~\ref{sec:evaluation} describes our
qualitative evaluation protocol and results, and Section~\ref{sec:discussion}
contextualizes these results within the overarching research framework.


\section{Game Design}
\label{sec:design}

\textit{Kaleb will provide text here describing the game design}

\section{Evaluation}
\label{sec:evaluation}

\subsection{Methods}

To answer our research questions requires building an understanding of
the player from descriptions and observations, and so we follow a
qualitative methodology~\citep[see][for example]{Stake2010}.  The
paucity of cybersecurity educational game assessments leaves almost no
grounds for forming the reliable and valid instruments necessary for
quantitative methods. Put another way, we could not know whether what
we measured would actually contribute to answering our research
questions.  \citeauthor{Cohen1994}'s classic essay highlights several
methodological problems with applying statistical methods to human
behavior, not the least of which is the common confusion of
\textit{probability of data} with \textit{probability of hypothesis
  correctness}~\citep{Cohen1994}.
Qualitative research methods, by contrast, deal with 
observations and descriptions, meeting the subjects within their
complex cultural contexts, and seeking understanding of phenomena
that are not directly measurable. 

Our research protocol was approved by Ball State University
Institutional Review Board~(835177-1). 
Our subject pool consisted of thirteen youth, ages ten to fifteen,
with eight females and five males. All of the subjects were
white American students living in Indiana. 
They came from families who responded to a call for participation that
was sent to Ball State University mailing lists and shared via social
media. This group is a convenience sample of those who responded to the
call, chosen for their ability to schedule sessions and with an emphasis
on female participants. The subjects include a mixture of homeschool
and public school attendees.

\begin{table}
\centering
\begin{tabular}{llr}
\textbf{Subject ID} & \textbf{Gender} & \textbf{Age}\\
1 & Female & 14\\
2 & Male & 10\\
3 & Female & 12\\
4 & Female & 11\\
5 & Male & 12\\
6 & Male & 15\\
7 & Female & 12\\
8 & Male & \textbf{??}\\
9 & Female & \textbf{??}\\
10 & Female & \textbf{??}\\
11 & Male & 15\\
12 & Female & 10\\
13 & Female & 10\\
\end{tabular}
\caption{Research Subjects}
\end{table}

Each session consisted of a semi-structured interview combined with
playtesting~(Appendix~\ref{appendix:interview-protocol}.
Interview audio was digitally recorded, and these were transcribed
into 918 paragraph-separated units. 
The investigators also 
produced~27 paragraphs of analytic memos,
% PVG 1140, KCS 229
consisting of~1378 words.
Gameplay behavior was automatically tracked using integrated logging, 
producing~16 logs consisting of~326 discrete game events.

Interview data were coded using techniques inspired by
\citet{Saldana2009}. The two researchers conducted three phases 
of coding before comparing results in order to ensure inter-rater
reliability. The first round consisted of open coding, and these
codes were refined in the second round. 
Appendix~\ref{appendix:phase-2} lists our second phase codes,
consolidated and standardized between researchers.
The third round consisted of
identifying categories, which represented the concepts that emerged
out of multiple second-round codes; the categories are listed in
Appendix~\ref{appendix:categories}.

\subsection*{Findings}

The five themes that emerged from our analysis are listed
in Table~\ref{tab:themes}. Each of these is described in more detail
below.

\begin{table}
\begin{itemize}
\item[] Surprise at the scale and persistence of cyberattacks
\item[] Positive experiences influence perspectives about career prospects
\item[] Mixed views on the role of education and degree toward career goals
\item[] Diverse opinions about developers' appearances and interests
\item[] Personal experiences strongly influenced their ability to learn from the game
\end{itemize}
\caption{Themes}
\label{tab:themes}
\end{table}

\newcommand{\theme}[1]{\subsubsection*{#1}}

\theme{Different modes of character empathy}

``\ldots{} with the hacker, they needed somebody to fix it, so
I sent the last one because she was good at fixing that kind
of thing. So, it just let me know which one would be
best for each job.'' 8-4

Why so quick to send Jerry to investigate the injection attack?
``He seemed like the most well-trained out of all of them,
and the girl---I forget her name---she seemed more like someone
who should be with, like, social, and not developer,
but Jerry seemed like the better person for the job of that kind of thing.''
8-7

Why did you like Janine so much?
``From her description, she seemed really on it. She seemed really cool.
I liked her a lot.'' 9-2

Why did you like Nancy?
``She looked really sassy. And she looked like me before
I went insane, 'cause I used to have long blond hair and that was
normal. It was gross. But she was really sassy, so I liked her.
`Sassy,' like, not really sassy but, like, `cool.' Like, she looked hip
and with the kids.'' 9-2

What did you think about the characters, after reading their
bios several times?
``I thought there was probably a pretty balanced team, because you had
someone who majored in three of the important categories,
such as computer engineering.
I can't remember what the one in the middle was, and
then there was another one who was good at working with people and
finding out what they want. That's why I chose here to do the---oh no,
I chose the engineering one to teach people how to do the password thing because
he know how to make a safe password, and then I would have chosen her to
find out what people liked\ldots{}.'' 9-4


\theme{Surprise at the scale and persistence of cyberattacks}

There was less clarity about the actual motives of the hackers.
8-2: ``Maybe they wanted to call the game their own or something?''



\theme{Positive experiences influence perspectives about career prospects}

% How do you imagine these people got jobs?
``\ldots{} their teachers saw their ability to do something
like that in an earlier stage, so when they were able to get
the classes in high school and through college to be able to get better
at that.'' 8-4

``I could report it to the police, or I could task someone to try to
figure it out. I'll probably just task somebody to figure it out for now
because, I don't really feel like getting mixed up with the police,
so I'll task [someone] to do it.'' 8-5


``No, I probably shouldn't have. I probably should have let
the police handle it instead of taking it into my own hands.'' 9-3

\theme{Mixed views on the role of education and degree toward career goals}

Reputation as an asset
``Well, somebody would hear about you, and you probably need to get
a degree, or be an entrepreneur and start your own app-making business.''8-5

Regarding school,
``I know I have to do it, but it's not very fun.'' 8-6

Why choose CS in child-to-work event?
``It seems like making an app would be more computers and that stuff than
anything else.'' 8-6

Where do apps come from?
``Anybody that makes them. \ldots{} If somebody decides they want to make 
an app, and they know how, they sit down and do it. If they don't know how,
and they still want to do it, then depending on how complex the app is they 
either go to try to figure it out at some random source, or if it's a bigger,
more complex project, they go take classes for it, and then they make the app.'' 8-6

What is the career path of a developer?
``There's probably a lot of mathematics and, you know, how you have to
know how games work or what other people did to make the games.'' 9-4


\theme{Diverse opinions about developers' appearances and interests}

Most of the players claimed to have never thought about the physical
characteristics of software developers, but among those that did,
there was great variance.  Subjects 6~and 13 reported developers to be
``geeky,'' which matched our expectations based on earlier work on
perceptions of engineers (\citet{Margolis2003}, for example). Neither
subject elaborated on what characterized geekiness, and we recognize
that there are potential misunderstandings in leaving the terms used
by teenagers up to the interpretation of adults.  
By contrast,
subject~10 stated that app developers look ``like normal
people\ldots{} like anybody.''
Subject~10 went on to say that the characters in the game
``didn't look like smart people---they looked like normal people.''

We did not expect the responses that painted app developers as something
attractive and aspirational. Subject~7 describes them as ``17 to sort of
early 20s in age,'' only about five to ten years older than herself.
Subject~9 was amused by Ivar's eccentric appearance, saying, ``He looks
kinda creepy, and I love it. I like his creepy long hair with a bald head.''
This was particularly amusing to us as Computer Scientists, given that
Ivar was modeled after Bjarne Stoustrup, but we did not bring this up
in the interview.

Subject~4 brings up a distinction between what she calls ``designers''
and ``producers,'' which we believe to correspond to the business functions
of application or interface design and programming. 
When asked what app developers look like, she puts it like this:
\begin{quote}
Well, I'm pretty sure [they look] like people, they're like artists
and stuff, \ldots{} I picture the people who \textit{design} it would
be kind of like artists, kind of, messy kind of regular people, and the
people who publish it would be kind of like, \textit{official}.
\end{quote}
It is worth noting that subject~4 was one of several students who had
completed a business information technology course in the local public
school, which we know to have included an introduction to programming
and which we believe to have informed students about some
specializations in the field.

Subject~6 is ``not a big fan of school,'' but he does
like \textit{Minecraft}. When asked about the appearance of \textit{Minecraft}'s
developers, he responds, ``Probably most of them are not American,
probably a lot of them are from Japan. It's a Japanese company, isn't it?''
While other subjects who brought up \textit{Minecraft} seemed to know something
about its Swedish creator, Markus Persson (``notch''), subject~6 held
an incorrect belief that it was created by Japanese developers. 
We suspect that this belief may be due to the relatively high number
of Japanese game development companies, this does not come up in the interview.

There was only one instance of overt sexism in the data:
when asked why he sent Esteban to the security conference over
Janine or Melissa, subject~2 responded cheekily, ``He's the guy.''
No other subject referenced developers' gender in their discussion
of their perceptions, backgrounds, interests, or educational preparation.
However, we recognize that gender and racial prejudices may be
subconscious or intentionally hidden from subjects' responses.


\theme{Right/Wrong, constructivism?}

``I knew some of them would hurt me no matter what I did so [I chose] the
best possible answers. And some of them, I really didn't know what to do,
so I just guessed.'' 9-4


\theme{Personal experiences strongly influenced their ability to learn from the game}

8-5 ``I actually kind of liked it. I liked the way you have to try and run your company---I just kind of like that type of thing. I don't know about other
kids\ldots{}''

``She has a bachelors of English. Oh, my mom went to Georgetown!'' 9-2

``Nancy has a popular podcast about being a woman in technology. Oh, that's cool!'' 9-2

(Not realizing she gets all the employees, she responds to reading Janine's bio early in the game.)
``I think I'm gonna go with Janine, because she sounds really cool.'' 9-2

\textit{Probably move this to a new theme}
This subject played the game twice and chose two different options in
response to the option for public disclosure.
``See, last time I was like, `Oh, I'll let them know,' and everyone
got mad. So I thought I'd just keep it a secret.
But then when I was exposed, everyone was like, `Why didn't you tell us that?!'
They all got really mad and left me, which I think is what made my exposure
go up. Last time they didn't like that; this time, they didn't like it more.''
9-2

\section{Discussion}
\label{sec:discussion}

\textit{Vocabulary and abstractions in game design: C/I/A, in-game text (press release). Authentic text vs. non-authentic. Role of playtesting.}

``Are exploits and features good or bad, what are these?'' 8-5

\textit{Randomness was not necessary to our goals}

\section{Conclusions and Future Work}

\section{Acknowledgments}

This research is based upon work supported by the National Science
Foundation under Grant No.~0968959 and by the Ball State University
Honors College.


\bibliographystyle{plainnat}
\bibliography{references}

\clearpage
\appendix
\section{Semi-structured interview protocol}
\label{appendix:interview-protocol}

\subsection*{Before playing the game}

\begin{enumerate}
\item What is your favorite subject in school?
\item What is your favorite app?
\item Who are the people who make [favorite app]?
\begin{enumerate}
\item What do they look like?
\item Did they go to college? What did they study?
\item What were their favorite subjects in school?
\end{enumerate}
\end{enumerate}

\subsection*{After playing the game}

\begin{enumerate}
\item Did anything surprise you?
\item What kinds of decisions did you make in the game? How did you make those decisions?
\item Where do apps come from?
\item Do you think you could get a job making apps? Why or why not?
\item How does someone get a job making apps?
\item Do you have any questions?
\end{enumerate}

\clearpage
\section{Second phase codes}
\label{appendix:phase-2}
\begin{multicols}{2}
\begin{verbatim}
academics: computer science
academics: science-engineering
academics: dislike

apps: communication
apps: games
apps: productivity
apps: for phones
apps: creative
apps: music

characters: pragmatic
characters: empathic

culture: male domination
culture: smart people look different

developers-like: math
developers-like: english
developers-like: art + stem
developers-like: technology
developers-like: science
developers-like: art
developers-like: design
developers-like: programming
developers-like: animation
developers-like: fundamentals
developers-like: computer science
developers-like: engineering

developers: not considered
developers: smart
developers: work hard
developers: crazy

developer-appearance: artists
developer-appearance: nonconformist
developer-appearance: geeky
developer-appearance: young
developer-appearance: smart and techy
developer-appearance: normal

development: difficulty
development: scale
development: coding
development: teams
development: testing
development: time
development: design separate from programming
development: entrepreneurship
development: anybody can do it

distribution: distinction of sources
distribution: companies as gatekeepers
distribution: appstore

economy: demand for apps
economy: jobs

education: degree has value
education: degree not necessary
education: degree as career
education: development classes
education: online resources
education: multiple paths

experience: scratch
experience: school
experience: code.org
experience: gamesalad

self-efficacy: technology requisite

game design: role-playing
game design: one right answer
game design: genre preferences
game design: empathy
game design: strategy

interest: problem-solving
interest: Minecraft
interest: creativity
interest: english
interest: science
interest: math
interest: writing
interest: terraria
interest: art
interest: band
interest: sketchbook pro
interest: history
interest: facebook
interest: wattpad
interest: gym
interest: clash of clans

learning: working hard
learning: improvement
learning: feedback loops

ssg: usability
ssg: vocabulary
ssg: math

security: disclosure
security: hackers
security: piracy
security: responsibility
security: pragmatism
security: encryption
security: confidentiality
security: number of exploits
security: strong passwords
\end{verbatim}
\end{multicols}

\clearpage
\section{Categories}
\label{appendix:categories}
\begin{itemize}
\item[] Computer science as the domain for app development
\item[] Character-based decisions
\item[] Academic vs non-academic learning
\item[] Incremental vs Entity theory of intelligence
\item[] Varied perspectives on developer appearance
\item[] Development is difficult
\item[] Companies as gatekeepers
\item[] Democracy of development
\item[] Economics and development
\item[] Degree as career
\item[] Positive educational experiences with coding
\item[] There is one right answer
\item[] Learning through practice
\item[] SSG issues
\item[] Surprise at hackers
\item[] Responsible disclosure
\item[] Protecting personal information
\item[] ``App'' means ``game''
\end{itemize}

\clearpage
\section{Characters in \textit{Social Startup Game}}

\newenvironment{character}[2]
{
\noindent\textbf{#1}
\begin{itemize*}
#2
\end{itemize*}
}
{
\bigskip
}

\begin{character}{Esteban Cortez}
{\item Bachelors in Computer Science, Ball State University}
Esteban worked in a factory until he was 33, then he went to college and decided to get involved in software development.
\end{character}

\begin{character}{Nancy Stevens}
{\item Bachelors in English, Georgetown University
 \item Masters in Computer Security, Purdue University}
Nancy has a popular podcast about being a woman in technology.
\end{character}

\begin{character}{Jerry Chen}
{\item Bachelors in Computer Science, University of Hong Kong}
Jerry interned at a local company in high school and has been working as a software developer ever since.
\end{character}

\begin{character}{Vani Mishra}
{\item Bachelors in Computer Engineering, Indian Institute of Science
 \item Masters in Software Engineering, Ball State University}
Vani was born in India and came to the United States for graduate school. She loves music, dancing, and PHP.
\end{character}

\begin{character}{Abdullah Nasr}
{\item Bachelors in Electrical Engineering, Iowa State University}
Abdullah used to work for a larger social media company, but he prefers the excitement of a small startup.
\end{character}

\begin{character}{Janine Palmer}
{\item Bachelors in Computer Science, Virginia Tech}
Janine is especially talented at meeting with customers and understanding what they want from a product.
\end{character}

\begin{character}{Melissa Barnard}
{\item Bachelors in Mathematics, Stanford University}
Melissa has always loved games and puzzles, but she especially loves bicycling and walking her German Shepherd.
\end{character}

\begin{character}{Ivar Johansen}
{\item Bachelors in Chemistry, Stockholm University
 \item Masters in Electrical Engineering, Uppsala University}
Ivar teaches kids how to build simple robots as a volunteer in a local school.
\end{character}

\begin{character}{Bruce Powers}
{\item Bachelors in Computer Science, Ball State University}
Bruce was recently married and enjoys going to the gym and watching documentaries about history.
\end{character}




\end{document}



